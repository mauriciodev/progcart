\documentclass{beamer}

\usepackage{aula}

\title{Ambiente de desenvolvimento}
\date{\today}

\begin{document}
	
\input{capa}

\begin{frame}{Python (em 2026?)}
	\begin{figure}
		\includegraphics[width=0.5\textwidth]{./introducao_figs/python2025.jpeg}
	\end{figure}
\end{frame}

\begin{frame}{Bibliotecas úteis}
	
	\begin{tabular}{p{4cm} p{7cm}}
		\toprule
		\textbf{Área} & \textbf{Bibliotecas Python} \\
		\midrule
		
		Matemática &
		NumPy, SciPy \\
		
		Interface gráfica (GUI) &
		PyQt5, Tkinter, Kivys \\
		
		Análises em SIG &
		PyQGIS, OGR, whitebox \\
		
		Dados raster geográficos &
		GDAL, Fiona, Rasterio \\
		
		Dataframes (tabelas) &
		Pandas, GeoPandas, Polars \\
		
		Gráficos científicos &
		Matplotlib, Plotly \\
		
		Backend Web &
		FastAPI, Flask, Django \\
		
		Backend com interface &
		Taipy, Streamlit, Dash \\
		
		Aprendizado de máquina &
		Scikit-learn, Keras, PyTorch, TensorFlow, JAX \\
		
		Aprendizado profundo para geoinformação & torchgeo, rastervision, geoai-py \\
		
		\bottomrule
	\end{tabular}
	
\end{frame}

\begin{frame}{Projetos}
	\begin{itemize}
		\item Plugin Hello Map
		\\ Carregar dados matriciais e vetoriais no QGIS (GDAL opcional)
		\item Abrir coordenadas da rede de monitoramento SIRGAS e converter para um arquivo vetorial
		\\ Baixar e manipular dados tabulados com Pandas e escrever com Geopandas.
		\item Processing Plugin de extrair cotas de um Modelo Digital de Superfícies
		\\ Manipular dados matriciais e vetoriais no QGIS
		\item Operações topológicas
		\\ Processing Plugin com model builder para resolver um problema de geoprocessamento.
		\item Opcionais (não valem nota)
		\begin{itemize}
			\item Classificação de imagens com RasterIO+Sklearn (XGBoost)
			\item Interfaces web com Streamlit			
			\item Geoagentes de IA
		\end{itemize}
		
	\end{itemize}
\end{frame}

\begin{frame}{Estrutura dos projetos}
	\begin{itemize}
		\item /docs
		\\ Diagrama de classes: Representa visualmente as principais classes que serão implementadas.
		\\ Fluxogramas: Representa uma visão geral de um procedimento importante.
		\item /src
		\\ Exemplos de código
		\item /tests 
		\\ Testes unitários: Verificam partes do código utilizando entradas e saídas controladas.
	\end{itemize}	
\end{frame}

\begin{frame}{Avaliação}
	\begin{itemize}
		\item 	6 – Implementação 
		\\ Entrega no github. Colocar o link para o repositório no Classroom.
		\\ 	Em pares.
		\item 	3 – Apresentação oral do código (no computador)
		\\ Alternado entre os dos dois membros do par.
		\\ Os alunos explicam o que cada parte do código está fazendo e as decisões tomadas. 
		\item 	1 – Inovação
		\\ 	Ponto para ideias interessantes que o par tiver. 
	\end{itemize}
\end{frame}

\begin{frame}{Ciclos de desenvolvimento em pares}
	\begin{itemize}
		\item Desenvolvimento ágil em ciclos de 2 semanas
		\item \url{https://ronjeffries.com/xprog/what-is-extreme-programming/}
		\item Programação em pares (Xtreme programming - XP)
	\end{itemize}
	\begin{figure}
		\includegraphics[width=\textwidth]{./introducao_figs/xp.png}
	\end{figure}
\end{frame}


\begin{frame}{Instalação do Python (com ambientes virtuais)}
Ambiente virtuais em Python: 
\begin{itemize}
	\item Isolamento de dependências entre projetos
	\item Evita conflitos de versões de bibliotecas
	\item Permite usar diferentes versões do Python no mesmo sistema
	\item Facilita a reprodução do ambiente em outros computadores
	\item Reduz riscos de quebrar aplicações existentes ao instalar novos pacotes
\end{itemize}
\textbf{Principal objetivo}: instalar GDAL e QGIS
\end{frame}

\begin{frame}{Instalação do Python (com ambientes virtuais)}
	Gerenciadores de ambientes:
	\begin{tabular}{p{2cm} p{2.5cm} p{2.8cm} p{3cm}}
		\toprule
		\textbf{Ferramenta} &
		\textbf{Descrição} &
		\textbf{Ambiente} &
		\textbf{Observações} \\
		\midrule
		
		pip + venv &
		Pacotes Python &
		Virtual no projeto &
		Padrão do Python \\ \midrule
		
		uv &
		pip + venv rápido &
		Ambientes e projetos &
		Alto desempenho \\ \midrule
		
		Conda &
		Pacotes diversos &
		Estrutura Conda &
		Ciência de dados \\ \midrule
		
		Pixi &
		Alternativa ao Conda &
		Estilo Conda &
		Substituto moderno \\
		
		\bottomrule
	\end{tabular}
	\begin{itemize}
		\item GDAL e QGIS: Pixi (recomendado) ou miniforge (conda).
		\item Deep learning: uv para pytorch, tensorflow, etc.
	\end{itemize}
\end{frame}



\begin{frame}{Pixi: visão geral}
	
	\textbf{Pixi} é uma ferramenta moderna para:
	\begin{itemize}
		\item Gerenciar \textbf{ambientes} de desenvolvimento
		\item Gerenciar \textbf{dependências} de projetos
		\item Reproduzir projetos de forma consistente
	\end{itemize}
	
	\vspace{0.3cm}
	
	\textbf{Principais características:}
	\begin{itemize}
		\item Baseado no ecossistema Conda
		\item Rápido e determinístico
		\item Um projeto = um ambiente
	\end{itemize}
	
	\vspace{0.3cm}
	
	\textbf{Ideia central:}  
	Cada projeto Python possui seu próprio ambiente isolado.
	
\end{frame}

\begin{frame}[fragile]{Pixi: Criando um projeto Pixi}
	
	\textbf{Passo 1: criar o projeto}
	
	\begin{minted}{bash}
pixi init meu_projeto
	\end{minted}
	
	Isso cria:
	\begin{itemize}
		\item Uma pasta do projeto
		\item O arquivo \texttt{pixi.toml}
	\end{itemize}
	
	\vspace{0.3cm}
	
	\textbf{Arquivo central: \texttt{pixi.toml}}
	\begin{itemize}
		\item Lista dependências
		\item Define versões
		\item Descreve o ambiente
	\end{itemize}
	
\end{frame}

\begin{frame}[fragile]{Pixi: Adicionando dependências}
	
	\textbf{Passo 2: adicionar pacotes ao projeto}
	
	\begin{minted}{bash}
pixi add python numpy pandas qgis
	\end{minted}
	
	O Pixi:
	\begin{itemize}
		\item Resolve dependências automaticamente
		\item Cria o ambiente isolado
		\item Atualiza o \texttt{pixi.toml}
	\end{itemize}
	
	\vspace{0.3cm}
	
	\textbf{Importante:}
	\begin{itemize}
		\item Não use \texttt{pip install} diretamente
		\item As dependências ficam documentadas no projeto
	\end{itemize}
	
\end{frame}

\begin{frame}[fragile]{Pixi: Executando código no ambiente}
	
	\textbf{Passo 3: usar o ambiente Pixi}
	
	\begin{minted}{bash}
pixi shell
	\end{minted}
	
	Entra em um shell com o ambiente ativo.
	
	\vspace{0.3cm}
	
	Ou execute comandos diretamente:
	
	\begin{minted}{bash}
pixi run python main.py
pixi run qgis
	\end{minted}
	
	\vspace{0.3cm}
	
	\textbf{Benefícios:}
	\begin{itemize}
		\item Mesmo ambiente para todos os usuários
		\item Menos erros de “funciona na minha máquina”
		\item Projeto mais organizado e reproduzível
	\end{itemize}
\end{frame}

\begin{frame}[fragile]{Inicialização de ambiente com UV}
	
	\textbf{O que é o UV?}  
	Gerenciador moderno de:
	\begin{itemize}
		\item ambientes virtuais
		\item dependências Python
	\end{itemize}
	
	\vspace{0.3cm}
	
	\textbf{Criando um novo projeto:}
	
	\begin{minted}{bash}
uv init projeto_geo
cd projeto_geo
	\end{minted}
	
	\vspace{0.3cm}
	
	\textbf{O que o UV cria automaticamente:}
	\begin{itemize}
		\item Arquivo \texttt{pyproject.toml}
		\item Ambiente virtual isolado
		\item Estrutura básica do projeto
	\end{itemize}
	
	\vspace{0.3cm}
	
	\textbf{Ativação do ambiente:}  
	O UV ativa o ambiente automaticamente ao executar comandos.
	
\end{frame}

\begin{frame}[fragile]{Instalação do GeoPandas com UV}
	
	\textbf{Instalando o GeoPandas:}
	
	\begin{minted}{bash}
uv add geopandas
	\end{minted}
	
	\vspace{0.3cm}
	
	\textbf{O que acontece nesse passo:}
	\begin{itemize}
		\item GeoPandas é adicionado ao \texttt{pyproject.toml}
		\item Dependências são resolvidas automaticamente
		\item O ambiente fica reproduzível
	\end{itemize}
	
	\vspace{0.3cm}
	
	\textbf{Testando a instalação:}
	
	\begin{minted}{python}
import geopandas as gpd
print(gpd.__version__)
	\end{minted}
	
	\vspace{0.3cm}
	
	\textbf{Boa prática:}  
	Sempre versionar o arquivo \texttt{pyproject.toml} no Git.
	
\end{frame}



\begin{frame}{Interface de desenvolvimento}
	\textbf{Integrated Development Environment (IDE)}
	\textbf{VSCodium}: \url{https://vscodium.com/}
	\begin{columns}[t]
		\begin{column}[t]{0.48\textwidth}

			Versão \textbf{código aberto} do VS Code.
			
			\textbf{Usaremos para:}
			\begin{itemize}
				\item Escrever código Python
				\item Depuração (Debug)
				\item Controle de versões de código (Git)
				\item Cadernos Jupyter
				\item Desenhar diagramas
				\begin{itemize}
					\item Fluxogramas
					\item UML: Classes
				\end{itemize}
			\end{itemize}
		\end{column}
		
		\begin{column}[t]{0.48\textwidth}
			\colimage{./introducao_figs/ides.png}
		\end{column}
	\end{columns}	
\end{frame}

\begin{frame}[fragile]{Interface de desenvolvimento}
	Caso ainda não tenha instalado, essa é a hora de instalar o VSCodium.
	Acessar \url{https://github.com/VSCodium/vscodium/releases}
	\begin{itemize}
		\item Windows: Baixar o arquivo VSCodiumUserSetup-x64-1.108.20787.exe e executar.
		\item Linux: Baixar o arquivo .deb e instalar com: 
		\\ \begin{minted}{bash}
dpkg -i codium_1.108.20787_amd64.deb
		\end{minted}
			
			
	\end{itemize}
\end{frame}

\begin{frame}{Copilotos de desenvolvimento}
	Origem: StackOverflow \url{https://stackoverflow.com/}
	
	\begin{itemize}
		
		\item ChatGPT \url{https://chatgpt.com/}
		\item Github Copilot \url{https://github.com/features/copilot}
		\item Gemini \url{https://gemini.google.com}
		\item Llama: \url{https://www.meta.ai/} (pesos e código abertos)
		\item \textbf{Instalação local}: Ollama \url{https://ollama.com/} 
	\end{itemize}
	Entre outros, alguns especialistas. 
	
	\espaco
	Não confie cegamente no trabalho de outra pessoa.  
	
	\espaco
	\centering
	\textbf{\Large{Muito menos de uma IA.}}
\end{frame}

\begin{frame}{VSCodium: Copilotos de desenvolvimento}
	Extensões disponíveis no VSCode/VSCodium:
	\begin{itemize}
		\item Github Copilot (nativo)
		\item Gemini 
		\item Codegpt
		\item Continue.dev + Ollama \textbf{Instalação local}		
	\end{itemize}
	\espaco
	Cuidados:
	\begin{itemize}
		\item Seu código é compartilhado com o dono do serviço (exceto se o serviço for seu).
		\item É fácil consumir todos os limites de uso das versões gratúitas.
	\end{itemize}
\end{frame}


\begin{frame}{VSCodium: Fluxograma}
	\begin{columns}[t]
		\begin{column}[t]{0.48\textwidth}
			
			\begin{itemize}
				\item Pasta: \url{projeto/docs/fluxograma.drawio}	
				\item Abrir com: \url{https://www.drawio.com/}			
				\item VSCodium Draw.io Integration (Author: Hediet)
				\item Tutorial de edição: \url{https://www.drawio.com/doc/getting-started-basic-flow-chart}
			\end{itemize}
		\end{column}
		
		\begin{column}[t]{0.48\textwidth}
			\colimage{./introducao_figs/fluxograma.png}
		\end{column}
	\end{columns}	
\end{frame}


\begin{frame}{VSCodium: Diagramas de classes (UML)}
	\begin{columns}[t]
		\begin{column}[t]{0.48\textwidth}
			\begin{itemize}
				\item Pasta: \url{projeto/docs/diagrama_de_classes.drawio}	
				\item Abrir com: \url{https://www.drawio.com/}			
				\item VSCodium Draw.io Integration (Author: Hediet)
				\item Tutorial de edição: \url{https://www.drawio.com/blog/uml-class-diagrams}s
			\end{itemize}
		\end{column}
		
		\begin{column}[t]{0.48\textwidth}
			\colimage{./introducao_figs/classes.png}
		\end{column}
	\end{columns}	
\end{frame}

\begin{frame}{VSCodium: Depuração (Debug)}
	
	O \textbf{debug} permite executar o programa passo a passo,
	inspecionando valores e o fluxo do código.
	
	\vspace{0.3cm}
	
	\textbf{Passos básicos:}
	\begin{itemize}
		\item Abrir o arquivo Python no VSCodium
		\item Definir \textbf{breakpoints} (clique à esquerda da linha)
		\item Iniciar o modo Debug
	\end{itemize}
	
	\vspace{0.3cm}
	
	\textbf{Durante o debug, é possível:}
	\begin{itemize}
		\item Executar linha a linha (\textit{Step Over / Step Into})
		\item Ver valores das variáveis em tempo real
		\item Avaliar expressões
		\item Identificar erros de lógica
	\end{itemize}
	
	\vspace{0.3cm}
	
	\textbf{Vantagem principal:}  
	Entender \textbf{como} o programa executa, não apenas o resultado final.
	
\end{frame}

\begin{frame}[fragile]{VSCodium: Debug com Pixi}
	
	\textbf{Objetivo:}  
	Depurar um código Python usando o \textbf{ambiente Pixi} dentro do VSCodium.
	
	\vspace{0.3cm}
	
	\textbf{Passo 1 — Abrir o projeto Pixi}
	\begin{itemize}
		\item Abra a pasta do projeto no VSCodium
		\item O arquivo \texttt{pixi.toml} deve estar na raiz
	\end{itemize}
	
	\vspace{0.3cm}
	
	\textbf{Passo 2 — Ativar o ambiente Pixi no terminal}
	\begin{minted}{bash}
pixi shell
	\end{minted}
	
	Isso garante que o Python e as dependências corretas estejam ativas.
	
	\vspace{0.3cm}
	
	\textbf{Passo 3 — Configurar o Debug}
	No menu \textit{Run and Debug}, escolha:
	\begin{itemize}
		\item \textbf{Python File}
		\item O VSCodium usará o Python do ambiente Pixi
	\end{itemize}
	
	\vspace{0.3cm}
	
	\textbf{Passo 4 — Depurar}
	\begin{itemize}
		\item Coloque breakpoints no código
		\item Inicie o Debug
		\item Use \textit{Step Over / Step Into}
	\end{itemize}
	
	\vspace{0.3cm}
	
	\textbf{Resultado:}  
	Código executado no \textbf{ambiente isolado do Pixi}, com depuração completa.
	
\end{frame}







%\begin{frame}{Cronograma de aulas}
%	
%\end{frame}


%\begin{frame}{titulo}
%	
%\end{frame}

%\begin{frame}[fragile]{Código Python}
%	\begin{minted}{python}
%	def soma(a, b):
%  return a + b
%	\end{minted}
%\end{frame}





\end{document}
