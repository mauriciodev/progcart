\documentclass{beamer}

\usepackage{aula}

\title{Ambiente de desenvolvimento}
\date{\today}

\begin{document}
	
\input{capa}


\begin{frame}{Instalação do Python (com ambientes virtuais)}
Ambiente virtuais em Python: 
\begin{itemize}
	\item Isolamento de dependências entre projetos
	\item Evita conflitos de versões de bibliotecas
	\item Permite usar diferentes versões do Python no mesmo sistema
	\item Facilita a reprodução do ambiente em outros computadores
	\item Reduz riscos de quebrar aplicações existentes ao instalar novos pacotes
\end{itemize}
\textbf{Principal objetivo}: instalar GDAL e QGIS
\end{frame}

\begin{frame}{Instalação do Python (com ambientes virtuais)}
	Gerenciadores de ambientes:
	\begin{tabular}{p{2cm} p{2.5cm} p{2.8cm} p{3cm}}
		\toprule
		\textbf{Ferramenta} &
		\textbf{Descrição} &
		\textbf{Ambiente} &
		\textbf{Observações} \\
		\midrule
		
		pip + venv &
		Pacotes Python &
		Virtual no projeto &
		Padrão do Python \\ \midrule
		
		uv &
		pip + venv rápido &
		Ambientes e projetos &
		Alto desempenho \\ \midrule
		
		Conda &
		Pacotes diversos &
		Estrutura Conda &
		Ciência de dados \\ \midrule
		
		Pixi &
		Alternativa ao Conda &
		Estilo Conda &
		Substituto moderno \\
		
		\bottomrule
	\end{tabular}
	\begin{itemize}
		\item GDAL e QGIS: Pixi (recomendado) ou miniforge (conda).
		\item Deep learning: uv para pytorch, tensorflow, etc.
	\end{itemize}
\end{frame}


\begin{frame}{Interface de desenvolvimento}
	\textbf{Integrated Development Environment (IDE)}
	\textbf{VSCodium}: \url{https://vscodium.com/}
	\begin{columns}[t]
		\begin{column}[t]{0.48\textwidth}

			Versão \textbf{código aberto} do VS Code.
			
			\textbf{Usaremos para:}
			\begin{itemize}
				\item Escrever código Python
				\item Depuração (Debug)
				\item Controle de versões de código (Git)
				\item Cadernos Jupyter
				\item Desenhar diagramas
				\begin{itemize}
					\item Fluxogramas
					\item UML: Classes
				\end{itemize}
			\end{itemize}
		\end{column}
		
		\begin{column}[t]{0.48\textwidth}
			\colimage{./introducao_figs/ides.png}
		\end{column}
	\end{columns}	
\end{frame}

\begin{frame}{Interface de desenvolvimento}
	Caso ainda não tenha instalado, essa é a hora de instalar o VSCodium
	\begin{itemize}
		\item Windows: winget VSCodium.VSCodium
		\item Linux: Baixar .deb em https://github.com/VSCodium/vscodium/releases
	\end{itemize}
\end{frame}

\begin{frame}{Copilotos de desenvolvimento}
	Origem: StackOverflow \url{https://stackoverflow.com/}
	
	\begin{itemize}
		
		\item ChatGPT \url{https://chatgpt.com/} (gratuito)
		\item Github Copilot \url{https://github.com/features/copilot} (gratuito)
		\item Gemini \url{https://gemini.google.com} (gratuito)	
		\item Llama: \url{https://www.meta.ai/} (gratuito + pesos e código abertos)
		\item \textbf{Instalação local}: OLLAMA \url{https://ollama.com/} 
	\end{itemize}
	Entre outros, alguns especialistas. 
	
	\espaco
	Não confie cegamente no trabalho de outra pessoa.  
	
	\espaco
	Muito menos de uma IA.
\end{frame}

\begin{frame}{Copilotos de desenvolvimento}
	Extensões disponíveis no VSCode/VSCodium:
	\begin{itemize}
		\item Github Copilot (nativo)
		\item Gemini 
		\item Continue.dev + Ollama \textbf{Instalação local}
		\item Codegpt
	\end{itemize}
	\espaco
	Cuidados:
	\begin{itemize}
		\item Seu código é compartilhado com o dono do serviço (exceto se o serviço for seu).
		\item É fácil consumir todos os limites de uso das versões gratúitas.
	\end{itemize}
\end{frame}


\begin{frame}{Fluxograma}
	\begin{columns}[t]
		\begin{column}[t]{0.48\textwidth}
			
			\begin{itemize}
				\item Pasta: \url{projeto/docs/fluxograma.drawio}	
				\item Abrir com: \url{https://www.drawio.com/}			
				\item VSCodium Draw.io Integration (Author: Hediet)
				\item Tutorial de edição: \url{https://www.drawio.com/doc/getting-started-basic-flow-chart}
			\end{itemize}
		\end{column}
		
		\begin{column}[t]{0.48\textwidth}
			\colimage{./introducao_figs/fluxograma.png}
		\end{column}
	\end{columns}	
\end{frame}


\begin{frame}{Diagramas de classes (UML)}
	\begin{columns}[t]
		\begin{column}[t]{0.48\textwidth}
			\begin{itemize}
				\item Pasta: \url{projeto/docs/diagrama_de_classes.drawio}	
				\item Abrir com: \url{https://www.drawio.com/}			
				\item VSCodium Draw.io Integration (Author: Hediet)
				\item Tutorial de edição: \url{https://www.drawio.com/blog/uml-class-diagrams}s
			\end{itemize}
		\end{column}
		
		\begin{column}[t]{0.48\textwidth}
			\colimage{./introducao_figs/classes.png}
		\end{column}
	\end{columns}	
\end{frame}

\begin{frame}{Ciclos de desenvolvimento}
\begin{itemize}
	\item Desenvolvimento ágil em ciclos de 2 semanas
	\item \url{https://ronjeffries.com/xprog/what-is-extreme-programming/}
	\item Programação em pares (Xtreme programming - XP)
\end{itemize}
\end{frame}


\begin{frame}{Projetos}
	\begin{itemize}
		\item Plugin Hello Map
		\item Carregar dados matriciais e vetoriais no QGIS (GDAL opcional)
		\item Abrir coordenadas da rede de monitoramento SIRGAS e converter para um arquivo vetorial
		\item Baixar e manipular dados tabulados com Pandas e escrever com Geopandas.
		\item Processing Plugin de extrair cotas de um Modelo Digital de Superfícies
		\item Manipular dados matriciais e vetoriais no QGIS
		\item Operações topológicas
		\item Processing Plugin com model builder para resolver um problema de geoprocessamento.
		\item (Opcional): RasterIO+Sklearn (XGBoost)
	\end{itemize}
\end{frame}



\begin{frame}{Estrutura dos projetos}
\begin{itemize}
	\item /docs
	\begin{itemize}
		\item Diagrama de classes: Representa visualmente as principais classes que serão implementadas.
		\item Fluxogramas: Representa uma visão geral de um procedimento importante.
	\end{itemize}
	\item /src
	\begin{itemize}
		\item Exemplos de código
	\end{itemize}
	\item /tests
	\begin{itemize}
		\item Testes unitários: Verificam partes do código utilizando entradas e saídas controladas.
	\end{itemize}
\end{itemize}	
\end{frame}


\begin{frame}{Avaliação}
	
\begin{itemize}
	\item 	6 – Implementação (entrega no classroom)
	\item 	Em pares.
	\item 	3 – Apresentação oral do código (no computador)
	\item 	Alternado entre os dos dois membros do par.
	\item 	1 – Inovação
	\item 	Ponto para ideias interessantes que o par tiver. 
\end{itemize}
\end{frame}

\begin{frame}{Cronograma de aulas}
	
\end{frame}

\begin{frame}{Bibliotecas úteis}

\begin{itemize}
	\item 	Matemática: Numpy, Scipy
	\item 	Interface gráfica (GUI): PyQt5, Tkinter
	\item 	Análises em SIG: PyQGIS, OGR
	\item 	Dados raster geográficos: GDAL, Fiona, Rasterio,
	\item 	Dataframes (tabelas): Pandas, Geopandas, Polars
	\item 	Gráficos científicos: Matplotlib, Plotly
	\item 	Backend Web: FastAPI, Flask, Django
	\item 	Backend com interface: Taipy, Streamlit
	\item 	Aprendizado de máquina: Sklearn, Keras, Torch, Tensorflow, Jax
\end{itemize}
\end{frame}


\begin{frame}{Prática pixi}
	
	\begin{itemize}
		\item Criar a pasta do projeto e acessar a pasta
		\item mkdir projeto0
		\item cd projecto0
		\item Inicializar o ambiente
		\item pixi init
		\item Ativar o ambiente criado:
		\item Instalando pacotes:
		\item pixi add qgis
	\end{itemize}
\end{frame}


\begin{frame}{Prática VSCodium}
\begin{itemize}
	\item Prática
	\item Fluxograma no VSCodium
	\item Debug no VSCodium
	\item Ctrl+Shift+P
	\item Select Python Interpreter
	\item Parar debugar: F5	
\end{itemize}
\end{frame}




%\begin{frame}{titulo}
%	
%\end{frame}

%\begin{frame}[fragile]{Código Python}
%	\begin{minted}{python}
%	def soma(a, b):
%  return a + b
%	\end{minted}
%\end{frame}





\end{document}
