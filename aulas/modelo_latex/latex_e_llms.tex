\documentclass{beamer}

%lembre-se de configurar o pdflatex com:
%pdflatex -shell-escape -synctex=1 -interaction=nonstopmode  -output-directory=build  %.tex

\usepackage{aula}
\usepackage{tikz}
\usepackage{tikz-3dplot}
\usetikzlibrary{calc}
%\usepackage{plantuml}
\usepackage{amsmath}
\usepackage{booktabs}

\title{Uso de LLMs para Produção de Conteúdo em Beamer}
\date{\today}

\begin{document}
    
\input{capa}


\begin{document}
    
    %------------------------------------------------
    
    %------------------------------------------------
    \begin{frame}{Introdução}
     \Large Modelos de linguagem são excelentes em traduzir conceitos para qualquer linguagem textual. \\
     
     Documentos baseados em texto simples são fáceis de editar e de submeter novamente ao modelo para correções. \\
     
    \end{frame}
    
    \begin{frame}{Introdução}
     LLMs podem auxiliar na:
     
     \begin{itemize}
        \item Geração automática de slides com beamer
        \item Código LaTeX estruturado
        \item Código Python com minted
        \item Diagramas UML com plantuml
        \item Estruturação de projetos (WBS) com plantuml
        \end{itemize}
    \end{frame}
    
    %------------------------------------------------
    \begin{frame}[fragile]{Código Python com minted}
     
    Ex.: Produza de 1 a 3 slides com latex beamer, demonstrando os principais conceitos do pacoted minted com exemplos em python.
     
     \begin{minted}[fontsize=\small, linenos]{python}
    import numpy as np
    
    def cobertura(h, theta):
    return 2*h*np.tan(theta/2)
    
    print(cobertura(30, 0.7))
     \end{minted}
     
     Requer compilação com:
     \texttt{pdflatex -shell-escape}
    \end{frame}
    
    %------------------------------------------------
    \begin{frame}{Figura Vetorial com TikZ}
     Ex.: Produza uma figura com latex TikZ que representa o plot da função X*x/2-1 no domínio de -2 a 2
     \begin{center}
         \begin{tikzpicture}[scale=0.8]
             \draw[->] (-2,0) -- (2,0);
             \draw[->] (0,-2) -- (0,2);
             \draw[domain=-2:2, smooth, variable=\x, blue] 
             plot ({\x}, {\x*\x/2 - 1});
         \end{tikzpicture}
     \end{center}
     
     Figura 100\% vetorial.
    \end{frame}
    
    \begin{frame}{Figura Vetorial com TikZ}
        Ex.: Produza uma figura tikz latex mostrando 4 drones sobre um campo de futebol. Represente a região imageada por cada drone. Utilize vista em perspectiva 3D.
        \begin{center}
            
    
    % Ângulos da câmera (elevação, azimute)
    \tdplotsetmaincoords{65}{115}
    
    \begin{tikzpicture}[tdplot_main_coords, scale=0.06]
        
        % ===============================
        % Parâmetros
        % ===============================
        \def\L{105}      % comprimento do campo
        \def\W{68}       % largura do campo
        \def\h{35}       % altitude dos drones
        \def\a{12}       % semi-largura do footprint
        \def\b{8}        % semi-altura do footprint
        
        \tikzset{
            drone/.style={
                circle,
                fill=black,
                inner sep=2pt
            }
        }
        
        % ===============================
        % Campo (plano z=0)
        % ===============================
        \draw[thick] (0,0,0) -- (\L,0,0) -- (\L,\W,0) -- (0,\W,0) -- cycle;
        
        % Linha central
        \draw (0.5*\L,0,0) -- (0.5*\L,\W,0);
        
        % Círculo central
        \draw (0.5*\L,0.5*\W,0) circle (9.15);
        
        % ===============================
        % Função para desenhar drone + pirâmide
        % ===============================
        
        \newcommand{\Drone}[4]{
            % #1 = x
            % #2 = y
            % #3 = cor
            % #4 = rótulo
            
            \coordinate (D) at (#1,#2,\h);
            
            % Footprint no solo
            \coordinate (F1) at (#1-\a,#2-\b,0);
            \coordinate (F2) at (#1+\a,#2-\b,0);
            \coordinate (F3) at (#1+\a,#2+\b,0);
            \coordinate (F4) at (#1-\a,#2+\b,0);
            
            % Desenho da região imageada
            \filldraw[#3!25, opacity=0.5] (F1)--(F2)--(F3)--(F4)--cycle;
            
            % Faces laterais (pirâmide)
            \filldraw[#3!20, opacity=0.4] (D)--(F1)--(F2)--cycle;
            \filldraw[#3!20, opacity=0.4] (D)--(F2)--(F3)--cycle;
            \filldraw[#3!20, opacity=0.4] (D)--(F3)--(F4)--cycle;
            \filldraw[#3!20, opacity=0.4] (D)--(F4)--(F1)--cycle;
            
            % Drone
            \node[drone] at (D) {};
            \node[above] at (D) {#4};
        }
        
        % ===============================
        % Quatro drones
        % ===============================
        
        \Drone{25}{50}{blue}{Drone 1}
        \Drone{80}{50}{red}{Drone 2}
        \Drone{25}{18}{green}{Drone 3}
        \Drone{80}{18}{purple}{Drone 4}
        
        % ===============================
        % Título
        % ===============================
        
        \node at (0.5*\L, \W+10, 0) {\Large Monitoramento 3D com 4 Drones};
        
    \end{tikzpicture}


        \end{center}
        
        Figura 100\% vetorial.
    \end{frame}
    
    
    %------------------------------------------------
    \begin{frame}[fragile]{UML gerado externamente com PlantUML}
    Ex.: Produza um diagrama plantuml representando as classes do código python abaixo: (inserir código)
    \begin{columns}[t]
        \begin{column}[t]{0.48\textwidth}
            \begin{minted}{text}
    @startuml
    class Drone {
    +id: int
    +altitude: float
    +capturarImagem()
    }
    
    class Controlador {
    +planejarMissao()
    }
    Drone --> Controlador
    @enduml
            \end{minted}
        \end{column}
        
        \begin{column}[t]{0.48\textwidth}
            
           \begin{center}
            \includegraphics[width=0.6\textwidth]{latex_e_llms_figs/uml_diagrama.png}
            \end{center}
        \end{column}
    \end{columns}                   
     Gerado via:
     \texttt{plantuml -tpdf uml\_diagrama.puml} \\ ou \url{https://www.plantuml.com/}
     
    \end{frame}
    
    %------------------------------------------------
    \begin{frame}[fragile]{WBS gerado externamente com PlantUML}
    \begin{columns}[t]
        \begin{column}[t]{0.48\textwidth}
            \begin{minted}{text}
    @startwbs
    * Projeto LLM + Beamer
    ** Modelagem
    *** Definir requisitos
    *** Definir arquitetura
    ** Implementação
    *** Slides
    *** Código
    *** Diagramas
    ** Validação
    *** Revisão técnica
    @endwbs
            \end{minted}
        \end{column}
        
        \begin{column}[t]{0.48\textwidth}
            \begin{center}
                \includegraphics[width=\linewidth]{latex_e_llms_figs/wbs_projeto.png}
            \end{center}
        \end{column}
    \end{columns}  
                     
    Gerado via:
    \texttt{plantuml -tpdf wbs\_projeto.puml} \\ ou \url{https://www.plantuml.com/}     
     
    \end{frame}
    
    %------------------------------------------------
    \begin{frame}{Geração de Equações}
     
     \[
     L = 2h \tan\left(\frac{\theta}{2}\right)
     \]
     
     \[
     J = \sum_{i=1}^{n} \| x_i - \hat{x}_i \|^2
     \]
     
     LLMs auxiliam na modelagem matemática formal.
    \end{frame}
    
    %------------------------------------------------
    \begin{frame}{7. Tabelas Técnicas}
     
     \begin{center}
         \begin{tabular}{lccc}
             \toprule
             Modelo & Parâmetros & Contexto & Uso \\
             \midrule
             GPT-4 & Alto & 128k & Pesquisa \\
             LLama & Médio & 32k & Local \\
             Mistral & Médio & 8k & Produção \\
             \bottomrule
         \end{tabular}
     \end{center}
     
    \end{frame}
    
    %------------------------------------------------
    \begin{frame}{8. Integração com Pipeline}
     
     Fluxo recomendado:
     
     \begin{enumerate}
         \item Gerar código com LLM
         \item Validar tecnicamente
         \item Gerar diagramas com PlantUML
         \item Compilar Beamer
         \item Versionar no Git
     \end{enumerate}
     
    \end{frame}
    
    %------------------------------------------------
    \begin{frame}{9. Automação com Makefile}
     
     Exemplo de fluxo automatizado:
     
     \begin{itemize}
         \item \texttt{make diagrams}
         \item \texttt{make slides}
         \item Integração CI/CD
     \end{itemize}
     
    \end{frame}
    
    %------------------------------------------------
    \begin{frame}{10. Conclusão}
     
     LLMs funcionam como:
     
     \begin{itemize}
         \item Assistente técnico
         \item Gerador de boilerplate
         \item Ferramenta de prototipação
         \item Copiloto acadêmico
     \end{itemize}
     
     Validação humana permanece essencial.
     
    \end{frame}
    
    %------------------------------------------------
\end{document} 
