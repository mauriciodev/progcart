
    
    % Ângulos da câmera (elevação, azimute)
    \tdplotsetmaincoords{65}{115}
    
    \begin{tikzpicture}[tdplot_main_coords, scale=0.06]
        
        % ===============================
        % Parâmetros
        % ===============================
        \def\L{105}      % comprimento do campo
        \def\W{68}       % largura do campo
        \def\h{35}       % altitude dos drones
        \def\a{12}       % semi-largura do footprint
        \def\b{8}        % semi-altura do footprint
        
        \tikzset{
            drone/.style={
                circle,
                fill=black,
                inner sep=2pt
            }
        }
        
        % ===============================
        % Campo (plano z=0)
        % ===============================
        \draw[thick] (0,0,0) -- (\L,0,0) -- (\L,\W,0) -- (0,\W,0) -- cycle;
        
        % Linha central
        \draw (0.5*\L,0,0) -- (0.5*\L,\W,0);
        
        % Círculo central
        \draw (0.5*\L,0.5*\W,0) circle (9.15);
        
        % ===============================
        % Função para desenhar drone + pirâmide
        % ===============================
        
        \newcommand{\Drone}[4]{
            % #1 = x
            % #2 = y
            % #3 = cor
            % #4 = rótulo
            
            \coordinate (D) at (#1,#2,\h);
            
            % Footprint no solo
            \coordinate (F1) at (#1-\a,#2-\b,0);
            \coordinate (F2) at (#1+\a,#2-\b,0);
            \coordinate (F3) at (#1+\a,#2+\b,0);
            \coordinate (F4) at (#1-\a,#2+\b,0);
            
            % Desenho da região imageada
            \filldraw[#3!25, opacity=0.5] (F1)--(F2)--(F3)--(F4)--cycle;
            
            % Faces laterais (pirâmide)
            \filldraw[#3!20, opacity=0.4] (D)--(F1)--(F2)--cycle;
            \filldraw[#3!20, opacity=0.4] (D)--(F2)--(F3)--cycle;
            \filldraw[#3!20, opacity=0.4] (D)--(F3)--(F4)--cycle;
            \filldraw[#3!20, opacity=0.4] (D)--(F4)--(F1)--cycle;
            
            % Drone
            \node[drone] at (D) {};
            \node[above] at (D) {#4};
        }
        
        % ===============================
        % Quatro drones
        % ===============================
        
        \Drone{25}{50}{blue}{Drone 1}
        \Drone{80}{50}{red}{Drone 2}
        \Drone{25}{18}{green}{Drone 3}
        \Drone{80}{18}{purple}{Drone 4}
        
        % ===============================
        % Título
        % ===============================
        
        \node at (0.5*\L, \W+10, 0) {\Large Monitoramento 3D com 4 Drones};
        
    \end{tikzpicture}

