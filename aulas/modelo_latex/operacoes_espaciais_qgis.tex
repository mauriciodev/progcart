\documentclass{beamer}

\usepackage{aula}

\title{Operações espaciais no QGIS}
\date{\today}

\begin{document}
	
\input{capa}

\begin{frame}[fragile]{Objetivos}
Ilustrar operações de geoprocessamento em dados matriciais e vetoriais.
% TODO: \usepackage{graphicx} required
\begin{figure}
	\centering
	\includegraphics[width=0.5\linewidth]{operacoes_espaciais_qgis_figs/toolbox}
	\caption{Exemplo: Ferramentas de geoprocessamento no QGIS (processing toolbox).}
	\label{fig:toolbox}
\end{figure}

\end{frame}

\begin{frame}{Relações espaciais: Pontos}
	\begin{columns}[t]
		\begin{column}[t]{0.48\textwidth}
			\begin{figure}
				\centering
				\includegraphics[width=0.7\linewidth]{operacoes_espaciais_qgis_figs/relacoes_ponto1}
			\end{figure}
		\end{column}
		
		\begin{column}[t]{0.48\textwidth}
			\begin{figure}
				\centering
				\includegraphics[width=0.7\linewidth]{operacoes_espaciais_qgis_figs/relacoes_ponto2}
			\end{figure}
		\end{column}
	\end{columns}	
	% TODO: \usepackage{graphicx} required
	\begin{figure}
		\centering
		\includegraphics[width=0.34\linewidth]{operacoes_espaciais_qgis_figs/relacoes_ponto3}
	\end{figure}
	Fonte: \href{https://inde.gov.br/pdf/ET-EDGV_versao_3.0_2018_05_20.pdf}{ET-EDGV 3.0 (link)} 
\end{frame}

\begin{frame}{Relações espaciais: Linhas}
	\begin{columns}[t]
		\begin{column}[t]{0.33\textwidth}
			\begin{figure}
				\centering
				\includegraphics[width=0.8\linewidth]{operacoes_espaciais_qgis_figs/relacoes_line1}
			\end{figure}
		\end{column}
		\begin{column}[t]{0.33\textwidth}
			\begin{figure}
				\centering
				\includegraphics[width=\linewidth]{operacoes_espaciais_qgis_figs/relacoes_line3}			
			\end{figure}
		\end{column}

		\begin{column}[t]{0.33\textwidth}
			\begin{figure}
				\centering
				\includegraphics[width=\linewidth]{operacoes_espaciais_qgis_figs/relacoes_line2}			
			\end{figure}
		\end{column}
	\end{columns}	
	\espaco
	
	Fonte: \href{https://inde.gov.br/pdf/ET-EDGV_versao_3.0_2018_05_20.pdf}{ET-EDGV 3.0 (link)} 
\end{frame}

\begin{frame}{Relações espaciais: Polígonos}
	\begin{columns}[t]
		\begin{column}[t]{0.33\textwidth}
			\begin{figure}
			\centering
			\includegraphics[width=0.8\linewidth]{operacoes_espaciais_qgis_figs/relacoes_poly1}

			\includegraphics[width=0.8\linewidth]{operacoes_espaciais_qgis_figs/relacoes_poly2}
			\end{figure}
		\end{column}
		
		\begin{column}[t]{0.66\textwidth}
			\begin{figure}
				\centering
				\includegraphics[width=\linewidth]{operacoes_espaciais_qgis_figs/relacoes_poly3}
				
			\end{figure}
		\end{column}
	\end{columns}	
	\espaco
	
	Fonte: \href{https://inde.gov.br/pdf/ET-EDGV_versao_3.0_2018_05_20.pdf}{ET-EDGV 3.0 (link)} 
\end{frame}

\begin{frame}{Simple feature access}
	\begin{itemize}
		\item O padrão determina os tipos de geometrias e as operações entre elas.
		\item 	As operações são definidas principalmente para SQL (bancos relacionais).
		\item 	Acessem em https://www.ogc.org/standards/sfs/
		\item 	OpenGIS Implementation Specification for Geographic information – Simple feature access – Part 2: SQL option 
	\end{itemize}
\end{frame}

\begin{frame}{Operações entre dados matriciais e vetoriais}
	Existem várias operações entre dados matriciais e vetoriais.
	
	% TODO: \usepackage{graphicx} required
	\begin{figure}
		\centering
		\includegraphics[width=0.4\linewidth]{operacoes_espaciais_qgis_figs/operacoes_matriciais_vetoriais}
		\label{fig:operacoesmatriciaisvetoriais}
	\end{figure}
	
\end{frame}

\begin{frame}[fragile]{Execução de módulos do processing em Python}
	\begin{minted}{python}
from qgis import processing
result = processing.run(
"native:buffer",
{
	'INPUT': layer,
	'OUTPUT': 'memory:'
},
context,
feedback
)
QgsProject.instance().addMapLayer(result['OUTPUT'])
	\end{minted}
	\begin{minted}{python}
import processing
processing.run(
"native:extractbyextent",
{
	'INPUT':'C:/Users/MDT-NOT/Desktop/Programação Aplicada/Aulas/poligonos.gpkg',
	'EXTENT':'-54.832642739,-54.570154142,-29.489490136,-29.270209539 [EPSG:4326]',
	'CLIP':True,
	'OUTPUT':'TEMPORARY_OUTPUT'
}
)
	\end{minted}

\end{frame}

\begin{frame}{Consultas Espaciais no Processing: Seleção}
	content...
\end{frame}

\begin{frame}{Consultas Espaciais no Processing: Extração}
	content...
\end{frame}


\begin{frame}[fragile]{Consultas Espaciais no PyQGIS}
	\begin{minted}{python}
# polygon_geometry contains a complex polygon, with many vertices
polygon_geometry = QgsGeometry.fromWkt('Polygon((...))')


# now we are ready to quickly test intersection against many other objects
for feature in my_layer.getFeatures():
	feature_geometry = feature.geometry()
	# test whether the feature's geometry intersects our original complex polygon
	if polygon_geometry.intersects(feature_geometry):
		print('feature intersects the polygon!')	
	\end{minted}
\end{frame}

\begin{frame}[fragile]{Consultas Espaciais no PyQGIS}
	Geometry Engine:
	\begin{minted}{python}
# polygon_geometry contains a complex polygon, with many vertices
polygon_geometry = QgsGeometry.fromWkt('Polygon((...))')

# create a QgsGeometryEngine representation of the polygon
polygon_geometry_engine = QgsGeometry.createGeometryEngine(polygon_geometry.constGet())

# since we'll be performing many intersects tests, we can speed up these tests considerably
# by first "preparing" the geometry engine
polygon_geometry_engine.prepareGeometry()
# now we are ready to quickly test intersection against many other objects
for feature in my_layer.getFeatures():
feature_geometry = feature.geometry()
# test whether the feature's geometry intersects our original complex polygon
if polygon_geometry_engine.intersects(feature_geometry.constGet()):
print('feature intersects the polygon!')
	\end{minted}
\end{frame}

\begin{frame}[fragile]{Índices Espaciais (Spatial Index)}
	teste
\end{frame}



\begin{frame}[fragile]{Índices Espaciais: Criação no PyQGIS}
	\begin{minted}{python}
		processing.run(
		"native:createspatialindex",
		{'INPUT':inputLyr},
		feedback=feedback,
		context=context,
		is_child_algorithm=is_child_algorithm
		)
	\end{minted}
\end{frame}

\begin{frame}[fragile]{Índices Espaciais: Criação no PyQGIS}
	\begin{minted}{python}
		spatialIdx = QgsSpatialIndex()
		idDict = {}
		for feature in layer.getFeatures():
		idDict[feature.id()] = feature
		spatialIdx.addFeature(feature)
	\end{minted}
\end{frame}

\begin{frame}[fragile]{Índices Espaciais: Uso de índice espacial no PyQGIS}
	\begin{minted}{python}
for featA in layerA.getFeatures():
geomA = featA.geometry()
bbox = geomA.boundingBox()
for featB in layerB.getFeatures(bbox): # usa o índice espacial para fazer a consulta
geomB = featB.geometry()
if geomB.intersects(geomA):
print(f"Feicoes {featA.id()} e {featB.id()} se intersectam")

layer_B_id_dict = {}
layerB_spatial_idx = QgsSpatialIndex()
for featB in layerB.getFeatures():
layer_B_id_dict[featB.id()] = featB
layerB_spatial_idx.addFeature(featB)

for featA in layerA.getFeatures():
geomA = featA.geometry()
bbox = geomA.boundingBox()
for id in layerB_spatial_idx.intersects(bbox):
featB = layer_B_id_dict[id]
geomB = featB.geometry()
if geomB.intersects(geomA):
print(f"Feicoes {featA.id()} e {featB.id()} se intersectam")
	\end{minted}
\end{frame}

\begin{frame}[fragile]{Índices Espaciais: Uso de índice espacial no PyQGIS}
	\begin{enumerate}
		\item Buffer
		\item Intersecção
		\item União
		\item Diferença Simétrica
		\item Diferença
	\end{enumerate}
\end{frame}

\begin{frame}{Exercícios}
\begin{itemize}
	\item \href{https://docs.qgis.org/3.40/pt_BR/docs/training_manual/vector_analysis/basic_analysis.html}{Processamento de dados vetoriais (Link)}
	\item 	\href{https://docs.qgis.org/3.40/pt_BR/docs/training_manual/rasters/terrain_analysis.html }{Processamento de dados matriciais (Link)}
\end{itemize}
	
\end{frame}




\end{document}
