\documentclass{beamer}

\usepackage{aula}

\title{Operações espaciais no QGIS}
\date{\today}

\begin{document}
	
\input{capa}

\begin{frame}[fragile]{Objetivos}
Ilustrar operações de geoprocessamento em dados matriciais e vetoriais.
% TODO: \usepackage{graphicx} required
\begin{figure}
	\centering
	\includegraphics[width=0.5\linewidth]{operacoes_espaciais_qgis_figs/toolbox}
	\caption{Exemplo: Ferramentas de geoprocessamento no QGIS (processing toolbox).}
	\label{fig:toolbox}
\end{figure}

\end{frame}

\begin{frame}{Relações espaciais: Pontos}
	\begin{columns}[t]
		\begin{column}[t]{0.48\textwidth}
			\begin{figure}
				\centering
				\includegraphics[width=0.7\linewidth]{operacoes_espaciais_qgis_figs/relacoes_ponto1}
			\end{figure}
		\end{column}
		
		\begin{column}[t]{0.48\textwidth}
			\begin{figure}
				\centering
				\includegraphics[width=0.7\linewidth]{operacoes_espaciais_qgis_figs/relacoes_ponto2}
			\end{figure}
		\end{column}
	\end{columns}	
	% TODO: \usepackage{graphicx} required
	\begin{figure}
		\centering
		\includegraphics[width=0.34\linewidth]{operacoes_espaciais_qgis_figs/relacoes_ponto3}
	\end{figure}
	Fonte: \href{https://inde.gov.br/pdf/ET-EDGV_versao_3.0_2018_05_20.pdf}{ET-EDGV 3.0 (link)} 
\end{frame}

\begin{frame}{Relações espaciais: Linhas}
	\begin{columns}[t]
		\begin{column}[t]{0.33\textwidth}
			\begin{figure}
				\centering
				\includegraphics[width=0.8\linewidth]{operacoes_espaciais_qgis_figs/relacoes_line1}
			\end{figure}
		\end{column}
		\begin{column}[t]{0.33\textwidth}
			\begin{figure}
				\centering
				\includegraphics[width=\linewidth]{operacoes_espaciais_qgis_figs/relacoes_line3}			
			\end{figure}
		\end{column}

		\begin{column}[t]{0.33\textwidth}
			\begin{figure}
				\centering
				\includegraphics[width=\linewidth]{operacoes_espaciais_qgis_figs/relacoes_line2}			
			\end{figure}
		\end{column}
	\end{columns}	
	\espaco
	
	Fonte: \href{https://inde.gov.br/pdf/ET-EDGV_versao_3.0_2018_05_20.pdf}{ET-EDGV 3.0 (link)} 
\end{frame}

\begin{frame}{Relações espaciais: Polígonos}
	\begin{columns}[t]
		\begin{column}[t]{0.33\textwidth}
			\begin{figure}
			\centering
			\includegraphics[width=0.8\linewidth]{operacoes_espaciais_qgis_figs/relacoes_poly1}

			\includegraphics[width=0.8\linewidth]{operacoes_espaciais_qgis_figs/relacoes_poly2}
			\end{figure}
		\end{column}
		
		\begin{column}[t]{0.66\textwidth}
			\begin{figure}
				\centering
				\includegraphics[width=\linewidth]{operacoes_espaciais_qgis_figs/relacoes_poly3}
				
			\end{figure}
		\end{column}
	\end{columns}	
	\espaco
	
	Fonte: \href{https://inde.gov.br/pdf/ET-EDGV_versao_3.0_2018_05_20.pdf}{ET-EDGV 3.0 (link)} 
\end{frame}

\begin{frame}{Simple feature access}
	\begin{itemize}
		\item O padrão determina os tipos de geometrias e as operações entre elas.
		\item 	As operações são definidas principalmente para SQL (bancos relacionais).
		\item 	Acessem em https://www.ogc.org/standards/sfs/
		\item 	OpenGIS Implementation Specification for Geographic information – Simple feature access – Part 2: SQL option 
	\end{itemize}
\end{frame}

\begin{frame}{Operações entre dados matriciais e vetoriais}
	Existem várias operações entre dados matriciais e vetoriais.
	
	% TODO: \usepackage{graphicx} required
	\begin{figure}
		\centering
		\includegraphics[width=0.4\linewidth]{operacoes_espaciais_qgis_figs/operacoes_matriciais_vetoriais}
		\label{fig:operacoesmatriciaisvetoriais}
	\end{figure}
	
\end{frame}

\begin{frame}{Exercícios}
\begin{itemize}
	\item \href{https://docs.qgis.org/3.40/pt_BR/docs/training_manual/vector_analysis/basic_analysis.html}{Processamento de dados vetoriais (Link)}
	\item 	\href{https://docs.qgis.org/3.40/pt_BR/docs/training_manual/rasters/terrain_analysis.html }{Processamento de dados matriciais (Link)}
\end{itemize}
	
\end{frame}




\end{document}
