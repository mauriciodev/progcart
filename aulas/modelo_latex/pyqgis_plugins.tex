\documentclass{beamer}

\usepackage{aula}

\title{PyQGIS - plugins de interface}
\date{\today}

\begin{document}
	
\input{capa}

\begin{frame}{Plugin Builder: Passo a passo}
\begin{figure}
	\centering
	\includegraphics[width=0.5\linewidth]{pyqgis_plugins_figs/builder_menu}
\end{figure}
\begin{figure}
	\centering
	\includegraphics[width=0.7\linewidth]{pyqgis_plugins_figs/builder_install}
\end{figure}
\end{frame}

\begin{frame}{Plugin Builder: Passo a passo}
	\begin{figure}
		\centering
		\includegraphics[width=0.5\linewidth]{pyqgis_plugins_figs/builder_menu2}
	\end{figure}
	\begin{figure}
		\centering
		\includegraphics[width=0.7\linewidth]{pyqgis_plugins_figs/builder_wizard1}
	\end{figure}
\end{frame}

\begin{frame}{Plugin Builder: Passo a passo}
	\begin{columns}
		\begin{column}{0.48\linewidth}
			\begin{figure}
				\centering
				\includegraphics[width=\linewidth]{pyqgis_plugins_figs/builder_wizard2}
			\end{figure}
		\end{column}
	
		\begin{column}{0.48\linewidth}
			\begin{figure}
				\centering
				\includegraphics[width=\linewidth]{pyqgis_plugins_figs/builder_wizard3}
			\end{figure}	
		\end{column}
	\end{columns}
\end{frame}

\begin{frame}{Plugin Builder: Passo a passo}
	\begin{columns}
		\begin{column}{0.48\linewidth}
			\begin{figure}
				\centering
				\includegraphics[width=\linewidth]{pyqgis_plugins_figs/builder_wizard4}
			\end{figure}
		\end{column}
		
		\begin{column}{0.48\linewidth}
			\begin{figure}
				\centering
				\includegraphics[width=\linewidth]{pyqgis_plugins_figs/builder_wizard5}
			\end{figure}	
		\end{column}
	\end{columns}
\end{frame}

\begin{frame}{Plugin Builder: Onde salvar?}
\begin{itemize}
	\item 	Onde salvar?
	\item 	LINUX
	\item 	\url{/home/username/.local/share/QGIS/QGIS3/profiles/default/python/plugins/}
	\item 	WINDOWS
	\item 	\url{C:\\Users\\<user>\\AppData\\Roaming\\QGIS\\QGIS3\\profiles\\default\\python\\plugins\\}
\end{itemize}
\end{frame}

\begin{frame}{Criação de Plugin: Estrutura básica do plugin}
	\begin{itemize}
		\item \_\_init\_\_.py = The starting point of the plugin. It has to have the classFactory() method and may have any other initialisation code.
		\item mainPlugin.py = The main working code of the plugin. Contains all the information about the actions of the plugin and the main code.
		\item resources.qrc = The .xml document created by Qt Designer. Contains relative paths to resources of the forms.
		\item resources.py = The translation of the .qrc file described above to Python.
		\item form.ui = The GUI created by Qt Designer.
		\item form.py = The translation of the form.ui described above to Python.
		\item metadata.txt = Contains general info, version, name and some other metadata used by plugins website and plugin infrastructure.
	\end{itemize}
	\url{https://docs.qgis.org/3.34/en/docs/pyqgis_developer_cookbook/plugins/plugins.html}
\end{frame}



\end{document}
