\documentclass{beamer}

\usepackage{aula}

\title{Introdução a Algoritmos}
\date{\today}

\begin{document}
    
\input{capa}

\begin{frame}{Strands Agents}
    
    \textbf{Strands Agents} é um framework para:
    \begin{itemize}
        \item Construir agentes baseados em LLMs
        \item Orquestrar raciocínio, ferramentas e memória
        \item Criar aplicações de IA orientadas a tarefas
    \end{itemize}
    
    \vspace{0.3cm}
    
    \textbf{Ideia central:}  
    Um agente que pensa, decide e age de forma estruturada.
    
\end{frame}

\begin{frame}{Agentes de IA}
    
    Um \textbf{agente} é um sistema que:
    \begin{itemize}
        \item Recebe um objetivo
        \item Observa o contexto
        \item Decide ações
        \item Executa ferramentas
    \end{itemize}
    
    \vspace{0.3cm}
    
    \textbf{Diferença para um chatbot simples:}
    \begin{itemize}
        \item Chatbot: responde mensagens
        \item Agente: resolve tarefas
    \end{itemize}
    
\end{frame}

\begin{frame}{Por que Strands Agents?}
    
    Sem um framework:
    \begin{itemize}
        \item Lógica de decisão fica espalhada
        \item Código difícil de manter
        \item Pouca reutilização
    \end{itemize}
    
    \vspace{0.3cm}
    
    Strands fornece:
    \begin{itemize}
        \item Estrutura clara para agentes
        \item Separação de responsabilidades
        \item Orquestração padronizada
    \end{itemize}
    
\end{frame}

\begin{frame}{Componentes de um Strands Agent}
    
    Um agente é composto por:
    \begin{itemize}
        \item \textbf{Modelo (LLM)}: raciocínio e decisões
        \item \textbf{Ferramentas (Tools)}: ações externas
        \item \textbf{Memória}: contexto e histórico
        \item \textbf{Objetivo}: o que deve ser resolvido
    \end{itemize}
    
    \vspace{0.3cm}
    
    Cada componente tem um papel bem definido.
    
\end{frame}

\begin{frame}{Ferramentas em Strands Agents}
    
    Ferramentas permitem ao agente:
    \begin{itemize}
        \item Executar código
        \item Consultar APIs
        \item Acessar bancos de dados
        \item Ler e escrever arquivos
    \end{itemize}
    
    \vspace{0.3cm}
    
    \textbf{Importante:}
    \begin{itemize}
        \item O LLM decide \textit{quando} usar a ferramenta
        \item O código define \textit{o que} a ferramenta faz
    \end{itemize}
    
\end{frame}

\begin{frame}{Memória do agente}
    
    A memória permite:
    \begin{itemize}
        \item Manter contexto da conversa
        \item Lembrar decisões anteriores
        \item Trabalhar em tarefas longas
    \end{itemize}
    
    \vspace{0.3cm}
    
    Tipos comuns:
    \begin{itemize}
        \item Memória de curto prazo
        \item Memória persistente
    \end{itemize}
    
\end{frame}

\begin{frame}{Fluxo de execução}
    
    Fluxo típico de um Strands Agent:
    \begin{enumerate}
        \item Recebe um objetivo
        \item Analisa o contexto
        \item Planeja ações
        \item Executa ferramentas
        \item Avalia o resultado
    \end{enumerate}
    
    \vspace{0.3cm}
    
    Esse ciclo pode se repetir até o objetivo ser atingido.
    
\end{frame}

\begin{frame}{Casos de uso}
    
    Strands Agents podem ser usados para:
    \begin{itemize}
        \item Assistentes inteligentes
        \item Automação de tarefas
        \item Análise de dados guiada por IA
        \item Agentes de suporte técnico
    \end{itemize}
    
    \vspace{0.3cm}
    
    \textbf{Ideais para tarefas complexas e multi-etapas.}
    
\end{frame}

\begin{frame}{Resumo}
    
    \begin{itemize}
        \item Strands Agents estruturam agentes de IA
        \item Integram LLMs, ferramentas e memória
        \item Facilitam aplicações orientadas a objetivos
    \end{itemize}
    
    \vspace{0.3cm}
    
    \textbf{Mensagem-chave:}  
    Strands não é apenas IA conversacional —  
    é IA \textbf{agente e operacional}.
    
\end{frame}



\begin{frame}[fragile]{Exemplo — Estrutura de um agente}
    
    \textbf{Ideia:}  
    O agente recebe uma pergunta, decide o que fazer e executa ações.
    
    \begin{minted}{python}
class AgenteSimples:
    def __init__(self):
        self.memoria = {}

    def responder(self, pergunta: str) -> str:
        # Decide o que fazer com base na pergunta
        if self._eh_calculo(pergunta):
            return self._resolver_calculo(pergunta)
        else:
            return self._responder_texto(pergunta)
    \end{minted}
    
    \vspace{0.3cm}
    
    \textbf{Conceitos envolvidos:}
    \begin{itemize}
        \item Agente encapsula lógica de decisão
        \item Não executa tudo diretamente
    \end{itemize}
    
\end{frame}

\begin{frame}[fragile]{Exemplo — Decisão e ferramentas}
    
    \begin{minted}{python}
def _eh_calculo(self, pergunta: str) -> bool:
    return any(op in pergunta for op in ["+", "*", "/"])

def _resolver_calculo(self, pergunta: str) -> str:
    # Exemplo simplificado
    resultado = calcular(2, 3, "soma")
    registrar_log("Cálculo executado")
    return f"Resultado: {resultado}"

def _responder_texto(self, pergunta: str) -> str:
    resposta = buscar_informacao(pergunta)
    salvar_memoria("ultima_pergunta", pergunta)
    return resposta
    \end{minted}
    
    \vspace{0.3cm}
    
    \textbf{Fluxo do agente:}
    \begin{enumerate}
        \item Analisa a pergunta
        \item Decide usar (ou não) uma ferramenta
        \item Retorna a resposta
    \end{enumerate}
    
\end{frame}


\begin{frame}[fragile]{Ferramenta — Cálculo numérico}
    
    \textbf{Objetivo:}  
    Executar cálculos quando o agente precisar de resultados exatos.
    
    \begin{minted}{python}
def calcular(a: float, b: float, operacao: str) -> float:
    if operacao == "soma":
        return a + b
    elif operacao == "multiplicacao":
        return a * b
    elif operacao == "divisao":
        if b == 0:
            raise ValueError("Divisão por zero")
            return a / b
        else:
            raise ValueError("Operação inválida")
    \end{minted}
    
    \vspace{0.3cm}
    
    \textbf{Uso pelo agente:}
    \begin{itemize}
        \item Quando a pergunta envolver números
        \item Quando precisão for necessária
    \end{itemize}
    
\end{frame}

\begin{frame}[fragile]{Ferramenta — Busca de informações}
    
    \textbf{Objetivo:}  
    Simular uma busca de informações externas.
    
    \begin{minted}{python}
def buscar_informacao(tema: str) -> str:
    base_conhecimento = {
        "python": "Python é uma linguagem de programação.",
        "git": "Git é um sistema de controle de versão.",
        "github": "GitHub hospeda repositórios Git."
    }
    return base_conhecimento.get(
        tema.lower(),
        "Informação não encontrada."
    )
    \end{minted}
    
    \vspace{0.3cm}
    
    \textbf{Uso pelo agente:}
    \begin{itemize}
        \item Perguntas factuais
        \item Consultas repetíveis
    \end{itemize}
    
\end{frame}

\begin{frame}[fragile]{Ferramenta — Registro de ações}
    
    \textbf{Objetivo:}  
    Registrar ações executadas pelo agente.
    
    \begin{minted}{python}
def registrar_log(mensagem: str) -> None:
    with open("log_agente.txt", "a") as arquivo:
        arquivo.write(mensagem + "\n")
    \end{minted}
    
    \vspace{0.3cm}
    
    \textbf{Exemplos de uso:}
    \begin{itemize}
        \item Registrar decisões do agente
        \item Acompanhar execução
    \end{itemize}
    
    \vspace{0.3cm}
    
    \textbf{Importante:}  
    Ferramentas podem causar efeitos fora do agente.
    
\end{frame}

\begin{frame}[fragile]{Ferramenta — Memória simples}
    
    \textbf{Objetivo:}  
    Armazenar e recuperar informações entre interações.
    
    \begin{minted}{python}
memoria = {}

def salvar_memoria(chave: str, valor: str) -> None:
    memoria[chave] = valor

def ler_memoria(chave: str) -> str:
    return memoria.get(chave, "Nada armazenado")
    \end{minted}
    
    \vspace{0.3cm}
    
    \textbf{Uso pelo agente:}
    \begin{itemize}
        \item Lembrar perguntas anteriores
        \item Manter contexto
    \end{itemize}
    
\end{frame}

\begin{frame}{Exercício — Strands Agent básico}
    
    \textbf{Objetivo do exercício:}  
    Compreender como estruturar um agente usando Strands.
    
    \vspace{0.3cm}
    
    \textbf{Cenário:}  
    Você deve criar um agente que ajude um usuário a:
    \begin{itemize}
        \item Receber uma pergunta
        \item Decidir se precisa usar uma ferramenta
        \item Retornar uma resposta adequada
    \end{itemize}
    
    \vspace{0.3cm}
    
    \textbf{Tarefas:}
    \begin{enumerate}
        \item Defina o \textbf{objetivo} do agente
        \item Liste quais \textbf{ferramentas} ele pode usar
        \item Descreva como o agente decide usar uma ferramenta
    \end{enumerate}
    
    \vspace{0.3cm}
    
    \textbf{Ferramentas sugeridas:}
    \begin{itemize}
        \item Uma função para calcular valores numéricos
        \item Uma função para buscar informações simuladas
    \end{itemize}
    
\end{frame}

\begin{frame}{Exercício — Orientação}
    
    \textbf{Dicas:}
    \begin{itemize}
        \item Pense no agente como um \textbf{resolvedor de tarefas}
        \item Separe claramente:
        \begin{itemize}
            \item Raciocínio (LLM)
            \item Ação (ferramentas)
        \end{itemize}
        \item O agente não deve executar tudo sozinho
    \end{itemize}
    
    \vspace{0.3cm}
    
    \textbf{Pergunta para reflexão:}
    \begin{itemize}
        \item Em quais situações o agente deve usar ferramentas?
        \item Quando apenas responder com texto é suficiente?
    \end{itemize}
    
\end{frame}

\begin{frame}{Desafio extra — Strands Agent}
    
    \textbf{Desafio:}
    \begin{itemize}
        \item Adicione \textbf{memória} ao agente
        \item Faça com que ele:
        \begin{itemize}
            \item Lembre a última pergunta do usuário
            \item Use essa informação na próxima resposta
        \end{itemize}
    \end{itemize}
    
    \vspace{0.3cm}
    
    \textbf{Objetivo pedagógico:}  
    Entender a diferença entre respostas isoladas e agentes com contexto.
    
\end{frame}

\end{document}
