\documentclass{beamer}

\usepackage{aula}

\title{Geopandas com cadernos Jupyter com e Leafmap}
\date{\today}

\begin{document}
	
\input{capa}

\begin{frame}{Jupyter Notebooks}
	\centering
	\huge Jupyter Notebooks \\
	(Cadernos Jupyter)\\
	Laboratório de ciência de dados.
\end{frame}

\begin{frame}{Google Colab}
	\url{https://colab.research.google.com/}
\end{frame}

\begin{frame}[fragile]{Instalação do Jupyter com Pixi}
	
	\textbf{Contexto:}  
	Pixi gerencia ambientes e dependências de projetos Python.
	
	\vspace{0.3cm}
	
	\textbf{1. Criar um projeto Pixi}
	\begin{minted}{bash}
pixi init projeto-jupyter
cd projeto-jupyter
	\end{minted}
	
	\vspace{0.2cm}
	
	\textbf{2. Adicionar Jupyter ao ambiente}
	\begin{minted}{bash}
pixi add jupyter
	\end{minted}
	
	\vspace{0.2cm}
	
	\textbf{3. Executar o Jupyter Notebook}
	\begin{minted}{bash}
pixi run jupyter notebook
	\end{minted}
	
	\vspace{0.3cm}
	
	\textbf{Observações:}
	\begin{itemize}
		\item Ambiente isolado e reprodutível
		\item Dependências descritas no \texttt{pixi.toml}
	\end{itemize}
	
\end{frame}
\begin{frame}[fragile]{Instalação do Jupyter com uv}
	
	\textbf{Contexto:}  
	\texttt{uv} é um gerenciador rápido de ambientes e pacotes Python.
	
	\vspace{0.3cm}
	
	\textbf{1. Criar ambiente virtual}
	\begin{minted}{bash}
uv venv
	\end{minted}
	
	\vspace{0.2cm}
	
	\textbf{2. Ativar o ambiente}
	\begin{minted}{bash}
source .venv/bin/activate
	\end{minted}
	
	\vspace{0.2cm}
	
	\textbf{3. Instalar Jupyter}
	\begin{minted}{bash}
uv pip install jupyter
	\end{minted}
	
	\vspace{0.2cm}
	
	\textbf{4. Executar o Jupyter Notebook}
	\begin{minted}{bash}
jupyter notebook
	\end{minted}
	
	\vspace{0.3cm}
	
	\textbf{Observações:}
	\begin{itemize}
		\item Fluxo similar ao \texttt{pip + venv}
		\item Foco em desempenho
	\end{itemize}
	
\end{frame}

\begin{frame}{Pandas e Geopandas}
	\centering
	\huge (geo) Data frames
\end{frame}

\begin{frame}{Matplotlib no Jupyter}

\end{frame}

\begin{frame}{Visualização de geoinformação no Jupyter}

\end{frame}

\begin{frame}[fragile]{Exemplo — Leafmap em caderno Jupyter}
	
	\textbf{Contexto:}  
	\texttt{leafmap} permite criar mapas interativos usando Python,
	integrando Jupyter + Leaflet.
	
	\vspace{0.3cm}
	
	\begin{minted}{python}
import leafmap

# Criar um mapa interativo
m = leafmap.Map(center=[-15, -55], zoom=4)

# Adicionar camada base
m.add_basemap("OpenStreetMap")

# Exibir o mapa no Jupyter
m
	\end{minted}
	
	\vspace{0.3cm}
	
	\textbf{O que acontece no notebook:}
	\begin{itemize}
		\item O mapa é renderizado interativamente
		\item Zoom e navegação com o mouse
		\item Ideal para exploração espacial
	\end{itemize}
	
\end{frame}

\end{document}
