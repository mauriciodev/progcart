\documentclass{beamer}
\usepackage{aula}

\title{Estruturas de Dados Básicas em Python}
\date{\today}

\begin{document}
	
	\begin{frame}
		\titlepage
	\end{frame}
	
	\begin{frame}[fragile]{Listas}
		\begin{itemize}
			\item Coleção ordenada e mutável
			\item Permite elementos de diferentes tipos
		\end{itemize}
		\begin{minted}{python}
lista = [1, 2, 3]
lista.append(4)
print(lista[0])
		\end{minted}
	\end{frame}
	
	\begin{frame}[fragile]{Operações com Listas}
		\begin{minted}{python}
numeros = [10, 20, 30]
numeros.remove(20)
tamanho = len(numeros)
		\end{minted}
	\end{frame}
	
	\begin{frame}[fragile]{Matrizes (Listas de Listas)}
		\begin{minted}{python}
import numpy as np
matriz = np.array([
  [1, 2, 3],
  [4, 5, 6]
])
print(matriz[1][2])
		\end{minted}
	\end{frame}
	
	\begin{frame}[fragile]{Percorrendo Matrizes}
		\begin{minted}{python}
for i in range(len(matriz)):
	for j in range(len(matriz[i])):
		print(matriz[i][j])
		\end{minted}
	\end{frame}
	
	\begin{frame}[fragile]{Tuplas}
		\begin{itemize}
			\item Estrutura imutável
		\end{itemize}
		\begin{minted}{python}
ponto = (3, 4)
x, y = ponto
		\end{minted}
	\end{frame}
	
	\begin{frame}[fragile]{Dicionários}
		\begin{minted}{python}
aluno = {
	"nome": "Ana",
	"idade": 20
}
print(aluno["nome"])
		\end{minted}
	\end{frame}
	
	\begin{frame}[fragile]{Percorrendo Dicionários}
		\begin{minted}{python}
for chave, valor in aluno.items():
	print(chave, valor)
		\end{minted}
	\end{frame}
	
	\begin{frame}{Exercícios}
		\begin{itemize}
			\item Criar uma lista e calcular a média
			\item Somar todos os elementos de uma matriz
			\item Criar um dicionário de cadastro
		\end{itemize}
	\end{frame}
	
\end{document}