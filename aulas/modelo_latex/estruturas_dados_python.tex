\documentclass{beamer}
\usepackage{aula}

\title{Estruturas de Dados Básicas em Python}
\date{\today}

\begin{document}
	
	\begin{frame}
		\titlepage
	\end{frame}
	
	\begin{frame}{Objetivos}
	\begin{itemize}
		\item Aprender a formatar strings em Python.
		\item Aprender as estruturas de dados mais usadas.
	\end{itemize}
	\end{frame}
	
	\begin{frame}[fragile]{Formatação de strings em Python}
		
		Existem três formas principais de formatar strings em Python:
		
		\vspace{0.3cm}
		
		\textbf{1. Operador \% (legado – evitar)}
		\begin{minted}{python}
nome = "Ana"
idade = 20
msg = "Nome: %s, Idade: %d" % (nome, idade)
		\end{minted}
		
		\vspace{0.2cm}
		
		\textbf{2. Método \texttt{str.format()}}
		\begin{minted}{python}
msg = "Nome: {}, Idade: {}".format(nome, idade)
msg = "Nome: {n}, Idade: {i}".format(n=nome, i=idade)
		\end{minted}
		
		\vspace{0.2cm}
		
		\textbf{3. f-strings (recomendado, Python $\ge$ 3.6)}
		\begin{minted}{python}
msg = f"Nome: {nome}, Idade: {idade}"
		\end{minted}
		
		\textbf{Vantagens das f-strings:}
		\begin{itemize}
			\item Mais legíveis
			\item Mais concisas
			\item Permitem expressões: \texttt{f"\{x + y\}"}
		\end{itemize}
		
	\end{frame}

	
	
	\begin{frame}[fragile]{Listas em Python}
		
		\textbf{Listas} são estruturas que armazenam uma sequência de valores,
		podendo conter elementos de tipos diferentes.
		
		\vspace{0.3cm}
		
		\textbf{Criação de listas}
		\begin{minted}{python}
numeros = [1, 2, 3, 4]
nomes = ["Ana", "João", "Maria"]
mistura = [10, "texto", 3.14]
		\end{minted}
		
		\vspace{0.2cm}
		
		\textbf{Acesso aos elementos (indexação começa em 0)}
		\begin{minted}{python}
print(numeros[0])   # 1
print(nomes[2])     # Maria
print(numeros[-1])  # último elemento
		\end{minted}
		
		\vspace{0.2cm}
		
		\textbf{Operações comuns}
		\begin{minted}{python}
numeros.append(5)      # adiciona no final
numeros.remove(2)      # remove o valor 2
tamanho = len(numeros) # quantidade de elementos
		\end{minted}
		
	\end{frame}

	
	
\begin{frame}[fragile]{Matrizes com NumPy}
	
	O \textbf{NumPy} fornece o tipo \texttt{ndarray}, usado para trabalhar
	com vetores e matrizes de forma eficiente.
	
	\vspace{0.3cm}
	
	\textbf{Importação e criação}
	\begin{minted}{python}
import numpy as np

A = np.array([
	[1, 2, 3],
	[4, 5, 6]
])
	\end{minted}
	
	\vspace{0.2cm}
	
	\textbf{Dimensões e acesso}
	\begin{minted}{python}
print(A.shape)     # (2, 3)
print(A[0, 1])     # elemento da 1ª linha, 2ª coluna
	\end{minted}
	
	\vspace{0.2cm}
	
	\textbf{Operações comuns}
	\begin{minted}{python}
B = A * 2           # operação elemento a elemento
C = A.T             # transposta
	\end{minted}
	
	\vspace{0.2cm}
	

	
\end{frame}

	
	\begin{frame}[fragile]{Operações com Matrizes}
		\textbf{Percorrendo a matriz com for}
		\begin{minted}{python}
for i in range(len(matriz)):
	for j in range(len(matriz[i])):
		print(matriz[i][j])
		\end{minted}
		\textbf{Criação rápida}
		\begin{minted}{python}
Z = np.zeros((3, 3))
I = np.eye(3)
		\end{minted}
	\end{frame}
	
	\begin{frame}[fragile]{Tuplas}
		\begin{itemize}
			\item Estrutura imutável
		\end{itemize}
		\begin{minted}{python}
ponto = (3, 4)
x, y = ponto
		\end{minted}
	\end{frame}
	
\begin{frame}[fragile]{Dicionários em Python}
	
	\textbf{Dicionários} armazenam pares \textbf{chave → valor}, permitindo
	acesso rápido aos dados por meio da chave.
	
	\vspace{0.3cm}
	
	\textbf{Criação de dicionários}
	\begin{minted}{python}
aluno = {
	"nome": "Ana",
	"idade": 20,
	"curso": "Computação"
}
	\end{minted}
	
	\vspace{0.2cm}
	
	\textbf{Acesso aos valores}
	\begin{minted}{python}
print(aluno["nome"])    # Ana
print(aluno["idade"])   # 20
	\end{minted}
	
	\vspace{0.2cm}
	
	\textbf{Inserção e modificação}
	\begin{minted}{python}
aluno["matricula"] = "20231234"
aluno["idade"] = 21
	\end{minted}
	
	\vspace{0.2cm}
	
	\textbf{Operações comuns}
	\begin{minted}{python}
chaves = aluno.keys()
valores = aluno.values()
pares = aluno.items()
	\end{minted}
	
\end{frame}

	
	\begin{frame}{Exercícios}
		\begin{itemize}
			\item Criar uma lista e calcular a média
			\item Somar todos os elementos de uma matriz
			\item Criar um dicionário de que relaciona o nome do aluno à sua nota.
		\end{itemize}
	\end{frame}
	
\end{document}