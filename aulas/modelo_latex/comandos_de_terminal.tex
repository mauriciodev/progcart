\documentclass{beamer}

%lembre-se de configurar o pdflatex com:
%pdflatex -shell-escape -synctex=1 -interaction=nonstopmode  -output-directory=build  %.tex

\usepackage{aula}

\title{Comandos de terminal}
\date{\today}

\begin{document}
    
\input{capa}

\begin{frame}[fragile]{Tópicos a abordar}

\begin{itemize}
    \item sys.argv: Recebimento de parâmetros
    \item argparse: Recebimento de parâmetros
    \item os: manipulação de 
    \item shutil
    \item pathlib
    \item glob
    \item subprocess
    \item Hydra
\end{itemize}
\end{frame}

\begin{frame}{subprocess: Conceitos}
    
    \textbf{subprocess} permite executar comandos externos a partir do Python.
    
    \vspace{0.4cm}
    
    Principais usos:
    
    \begin{itemize}
        \item Executar scripts shell
        \item Chamar programas externos (GDAL, QGIS, etc.)
        \item Automatizar pipelines
    \end{itemize}
    
    \vspace{0.4cm}
    
    Função mais usada:
    
    \begin{itemize}
        \item \texttt{subprocess.run()}
    \end{itemize}
    
    \vspace{0.4cm}
    
    Boa prática:
    \begin{itemize}
        \item Passar argumentos como lista
        \item Evitar \texttt{shell=True} quando possível
    \end{itemize}
    
\end{frame}

\begin{frame}[fragile]{subprocess: Executando comando com argumentos}
    
    \begin{minted}{python}
import subprocess

comando = [
    "gdalinfo",
    "imagem.tif"
]

resultado = subprocess.run(
    comando,
    capture_output=True,
    text=True
)

print("STDOUT:")
print(resultado.stdout)

print("STDERR:")
print(resultado.stderr)
    \end{minted}
    
    \vspace{0.3cm}
    
    \textbf{Parâmetros importantes:}
    \begin{itemize}
        \item capture\_output=True
        \item text=True (retorna string)
    \end{itemize}
    
\end{frame}
\begin{frame}[fragile]{subprocess: Exemplo — Salvando stdout e stderr em log}
    
    \begin{minted}{python}
import subprocess

comando = ["ls", "-l", "/tmp"]

with open("execucao.log", "w") as log:
    processo = subprocess.run(
        comando,
        stdout=log,
        stderr=log,
        text=True
    )

print("Código de saída:", processo.returncode)
    \end{minted}
    
    \vspace{0.3cm}
    
    \textbf{O que acontece:}
    \begin{itemize}
        \item stdout → escrito no arquivo
        \item stderr → escrito no mesmo log
        \item returncode → indica sucesso (0)
    \end{itemize}
    
\end{frame}
\end{document}
