\documentclass{beamer}

\usepackage{aula}

\title{Git - Controle de versões de arquivos}
\date{\today}

\begin{document}
    
\input{capa}

\begin{frame}{GitHub}
    
    \textbf{GitHub} é uma plataforma online para:
    \begin{itemize}
        \item Hospedar repositórios Git
        \item Compartilhar código
        \item Colaborar em projetos
    \end{itemize}
    
    \vspace{0.3cm}
    
    \textbf{Em resumo:}  
    GitHub = Git + colaboração + nuvem
    
\end{frame}

\begin{frame}{GitHub e Git: qual a diferença?}
    
    \begin{itemize}
        \item \textbf{Git}
        \begin{itemize}
            \item Ferramenta local
            \item Controla versões do código
            \item Funciona sem internet
        \end{itemize}
        
        \vspace{0.3cm}
        
        \item \textbf{GitHub}
        \begin{itemize}
            \item Serviço online
            \item Hospeda repositórios Git
            \item Facilita colaboração
        \end{itemize}
    
        \vspace{0.3cm}
        \item \textbf{Repositórios Git+GitHub:}
        \begin{itemize}
            \item \textbf{Remotos}: repositórios conectados de controle de versão
            \item \textbf{Locais}: cópias locais do repositório
        \end{itemize}
    \end{itemize}
\end{frame}

\begin{frame}{Repositórios no GitHub}
    
    Um \textbf{repositório} é:
    \begin{itemize}
        \item Um projeto versionado com Git
        \item Armazenado online no GitHub
    \end{itemize}
    
    \vspace{0.3cm}
    
    Pode conter:
    \begin{itemize}
        \item Código-fonte
        \item Documentação
        \item Histórico de alterações
    \end{itemize}
    
    \vspace{0.3cm}
    
    Cada repositório possui:
    \begin{itemize}
        \item Um endereço (URL)
        \item Permissões de acesso
    \end{itemize}
    
\end{frame}

\begin{frame}{Ações comuns no GitHub}
    
    \begin{itemize}
        \item \textbf{Clone}: baixar o repositório para o computador
        \item \textbf{Push}: enviar commits para o GitHub
        \item \textbf{Pull}: baixar atualizações
        \item \textbf{Fork}: criar uma cópia do repositório
        \item \textbf{Pull Request}: propor alterações
    \end{itemize}
    
    \vspace{0.3cm}
    
    Essas ações permitem o trabalho colaborativo.
    
\end{frame}

\begin{frame}{Por que usar o GitHub?}
    
    \begin{itemize}
        \item Trabalhar em equipe
        \item Manter histórico do projeto
        \begin{itemize}
            \item \textbf{Lembrar o que você fez}
            \item \textbf{Achar onde foi que você quebrou o código}
        \end{itemize}
        \item Facilitar revisão de código
        \item Compartilhar projetos

    \end{itemize}
    
    \vspace{0.3cm}
    
    \textbf{Benefícios:}
    \begin{itemize}
        \item Transparência
        \item Organização
        \item Reprodutibilidade
    \end{itemize}
    
\end{frame}

\begin{frame}{GitHub}
    
    Cada aluno deve criar seu próprio usuário
    \begin{itemize}
        \item Acessar \url{https://github.com}
        \item Criar uma conta
    \end{itemize}
    
    \vspace{0.3cm}
    
    Essas credenciais serão usadas:
    \begin{itemize}
        \item No VSCodium ou GitHub Desktop
        \item Para baixar código
        \item Para submeter alterações
    \end{itemize}
    
\end{frame}

\begin{frame}{GitHub Web - Criando Repositório via Interface Web}
    \begin{columns}
        \begin{column}[t]{0.40\textwidth}
            Passos:
            \begin{enumerate}
                \item Acessar github.com
                \item Clicar em "New repository"
                \item Definir:
                \begin{itemize}
                    \item Nome
                    \item Descrição
                    \item Público ou Privado
                \end{itemize}
                \item (Opcional) Inicializar com README
                \item Criar repositório
            \end{enumerate}
        \end{column}
        \begin{column}[t]{0.60\textwidth}
            Após fazer o login, crie um repositório na sua conta.
            \begin{figure}
                \centering
                \includegraphics[width=\textwidth]{GIT_figs/new_repo}
            \end{figure}
        \end{column}
    \end{columns}
    
\end{frame}

\begin{frame}{GitHub Desktop — Login do usuário}
    
    \begin{columns}[T]
        % TEXTO (30%)
        \begin{column}{0.4\textwidth}
            \textbf{Passo 1 — Login}
            
            \begin{itemize}
                \item Abrir o GitHub Desktop
                \item File $\rightarrow$ Options $\rightarrow$ Accounts
                \item Fazer login com a conta GitHub
            \end{itemize}
            
            \vspace{0.3cm}
            Necessário para clonar e sincronizar repositórios.
        \end{column}
        
        % IMAGEM (70%)
        \begin{column}{0.7\textwidth}
            \centering
            \includegraphics[width=\linewidth]{GIT_figs/login.png}
        \end{column}
    \end{columns}
    
\end{frame}


\begin{frame}[fragile]{GitHub Desktop — Clonar repositório}
    \textbf{Passo 2 — Clonar}
    \begin{minted}{bash}
https://github.com/usuario/meu-repo.git
    \end{minted}
    \begin{columns}[T]
        \begin{column}{0.48\textwidth}
            \begin{figure}
                \centering
                \includegraphics[width=\textwidth]{GIT_figs/clone}
            \end{figure}
        \end{column}
        
        \begin{column}{0.48\textwidth}
            \begin{figure}
                \centering
                \includegraphics[width=\textwidth]{GIT_figs/clone1}
            \end{figure}
            
        \end{column}
    \end{columns}
    \vspace{0.3cm}
    O repositório é baixado apenas uma vez.
\end{frame}


\begin{frame}{GitHub Desktop — Ver modificações}
    
    \begin{columns}[T]
        \begin{column}{0.4\textwidth}
            \textbf{Passo 3 — Alterações}
            
            \begin{itemize}
                \item Editar arquivos no editor
                \item Salvar alterações
                \item Ver aba \textit{Changes}
            \end{itemize}
            
            \vspace{0.3cm}
            As diferenças são mostradas linha a linha.
        \end{column}
        
        \begin{column}{0.7\textwidth}

        \begin{figure}
            \centering
            \includegraphics[width=\textwidth]{GIT_figs/diff2}
        \end{figure}

        \end{column}
    \end{columns}
    
\end{frame}

\begin{frame}{GitHub Desktop — Commit}
    
    \begin{columns}[T]
        \begin{column}{0.4\textwidth}
            \textbf{Passo 4 — Commit}
            
            \begin{itemize}
                \item Selecionar arquivos
                \item Escrever mensagem
                \item Commit to main
            \end{itemize}
            
            \vspace{0.3cm}
            Commit registra mudanças localmente.
        \end{column}
        
        \begin{column}{0.7\textwidth}
            \begin{figure}
                \centering
                \includegraphics[width=\textwidth]{GIT_figs/commit}
            \end{figure}
        \end{column}
    \end{columns}
    
\end{frame}

\begin{frame}{GitHub Desktop — Push e Pull}
    
    \begin{columns}[T]
        \begin{column}{0.4\textwidth}
            \textbf{Passo 5 — Push}
            \begin{itemize}
                \item Envia commits ao GitHub
            \end{itemize}
            
            \textbf{Passo 6 — Pull}
            \begin{itemize}
                \item Baixa atualizações remotas
            \end{itemize}
            
            \vspace{0.3cm}
            Push = local → remoto  \\
            Pull = remoto → local
        \end{column}
        
        \begin{column}{0.7\textwidth}
        % TODO: \usepackage{graphicx} required
        \begin{figure}
            \centering
            \includegraphics[width=\textwidth]{GIT_figs/push}
            \caption{}
            \label{fig:push}
        \end{figure}

        \end{column}
    \end{columns}
    
\end{frame}



\begin{frame}[fragile]{Git no terminal - Instalação}
    
    \textbf{Windows (PowerShell):}
    \begin{minted}{bash}
winget install Git.Git
    \end{minted}
    
    \textbf{Linux (shell):}
    \begin{minted}{bash}
sudo apt install git
    \end{minted}
    
\end{frame}

\begin{frame}[fragile]{Git no terminal - Configuração}
    
    Após a instalação, configure suas credenciais:
    
    \begin{minted}{bash}
git config --global user.email "you@example.com"
git config --global user.name "Your Name"
    \end{minted}
    
    Essas informações identificam o autor dos commits.
    
\end{frame}

\begin{frame}[fragile]{Git no terminal - Clonando Repositório}
    Clonar repositório no modo somente leitura:
    \begin{minted}{bash}
git clone https://github.com/mauriciodev/progcart.git
    \end{minted}
        \vspace{0.3cm}
    Clonar repositório no modo leitura e escrita (via ssh):
    \begin{minted}{bash}
git clone git@github.com/mauriciodev/progcart.git
    \end{minted}
        \vspace{0.3cm}
    Estrutura criada:
    \begin{itemize}
        \item Diretório do projeto
        \item Pasta .git
    \end{itemize}
\end{frame}

\begin{frame}{Git no VSCodium}
    
    \textbf{Separação de arquivos:}
    \begin{itemize}
        \item Arquivos \textbf{tracked}: já registrados no Git
        \item Arquivos \textbf{untracked}: ainda não versionados
    \end{itemize}
    
    \vspace{0.3cm}
    
    Objetivo:
    \begin{itemize}
        \item Evitar versionar arquivos que não fazem parte do código-fonte
    \end{itemize}
    
    \vspace{0.3cm}
    
    \textbf{Atalho:}
    \begin{itemize}
        \item \texttt{Ctrl + Shift + P}
    \end{itemize}
    
\end{frame}

\begin{frame}{GitHub — Autorização da máquina}
    
    Para acessar o GitHub a partir de um computador:
    \begin{itemize}
        \item É necessário cadastrar a chave SSH da máquina
    \end{itemize}
    
    \vspace{0.3cm}
    
    \textbf{Linux/Windows: criar chave SSH}
    \begin{itemize}
        \item No terminal: \texttt{ssh-keygen -t ed25519}
        \item Abrir o arquivo \texttt{.pub} gerado na pasta .ssh do seu usuário.
        \item No GitHub acessar: Settings → SSH and GPG Keys → New SSH Key.
        \item Copiar o conteúdo do arquivo para o GitHub.
    \end{itemize}
    
    \vspace{0.2cm}
\end{frame}

\begin{frame}{Exercícios}
    
    \begin{enumerate}
        \item Fazer um \textbf{fork} do repositório da disciplina:
        \begin{itemize}
            \item \url{https://github.com/mauriciodev/progcart}
        \end{itemize}
        
        \item Criar um \textbf{clone} do repositório na sua máquina
        \item Verificar a timeline das versões
        \item Verificar o histórico de um arquivo
        \item Alterar um arquivo
        \item Fazer um \textbf{commit}
        \item Fazer um \textbf{push}
        \item Acessar o GitHub e verificar a ferramenta de \textbf{Pull Request}
    \end{enumerate}
    
\end{frame}

\begin{frame}[fragile]{Git no terminal - Criando Repositório Local}
    Caso o projeto já exista:
    
    \begin{minted}{bash}
git init
git add .
git commit -m "Initial commit"
    \end{minted}
    
    Adicionar remoto:
    
    \begin{minted}{bash}
git remote add origin https://github.com/usuario/meu-repo.git
git push -u origin main
    \end{minted}
\end{frame}

%
%\begin{frame}[fragile]{Código Python}
%   \begin{minted}{python}
%   def soma(a, b):
%  return a + b
%   \end{minted}
%\end{frame}
%
%\begin{frame}[fragile]{Código Python}
%   \begin{columns}[t]
%       \begin{column}[t]{0.48\textwidth}
%           
%           Versão \textbf{código aberto} do VS Code.
%           
%           \textbf{Usaremos para:}
%           \begin{itemize}
%               \item teste
%           \end{itemize}
%       \end{column}
%       
%       \begin{column}[t]{0.48\textwidth}
%           \colimage{./introducao_GIT_figs/ides.png}
%       \end{column}
%   \end{columns}   
%\end{frame}





\end{document}
